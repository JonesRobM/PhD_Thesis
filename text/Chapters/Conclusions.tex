\noindent\enquote{\itshape Veni, vidi, vici.}\bigbreak

\hfill Gaius Julius Caesar

\vspace*{0.05\textheight}

Metallic nanoparticles are complex objects whose behaviours diverge from that of their bulk counterparts. There has been great interest and fervent scientific efforts focused into the understanding, design, and realisation of these fascinating systems. In this thesis, we have primarily seen these systems as candidates for plasmon enhanced photo-catalysts. However; their utility is far wider than this single application and it is undeniable that they deserve the attention they receive.

At its heart, this thesis has asked the question: ``What is the best nano-architecture which enhances photo-catalytic performance?''. As we have determined through the course of the presented research, this is a multifaceted question with many subtleties embedded within. Whilst we have not necessarily reached a categorically specific answer to this question, we have certainly narrowed the sample space through our multi-scale modelling approach. Indeed, it is only right and proper that we have not yet answered this question as it is worth scholarly discussion and debate due to the inherent intractable nature of the problem. Moreover, that a definitive answer still eludes us provides ample incentive to continue in this line of research and continue to learn about these fascinating systems from many perspectives. Indeed, beyond the scope of incentivising the use of a multi-scale approach to decouple the extensive range of time and length scales, we have also continued to refer and defer to the experimental community for guidance in the synthesis and interrogation of nanoalloys for the purposes of catalysis. As is common in reverse, that experimental communities require the input and discussion available within the theoretical and modelling communities. That is to say, in addition to requiring a multi-scale approach - so too do we require an interdisciplinary approach to sufficiently address the manifold areas of scholarly intrigue.

We have discussed at great length the computational and analytic techniques currently adopted across a range of communities to study localised surface plasmons on metallic nanoparticles, the catalytic performance of these nanoparticles, and indeed their stability. We have discussed two regimes in which one may model such systems, being at the classical level and in the quantum mechanical regime. We have demonstrated the necessity of having a multi-scale modelling approach spanning targeting specifically each of these regimes. This being a consequence of the fact that the necessary size and time scales to understand the stability and time dependent evolution of a composite structure are presently only accessible via contemporary classical molecular dynamics methods. Conversely, only by considering electrons explicitly as fundamentally quantum mechanical in nature may we strive to accurately describe the optical properties of ultra-small systems whose finite size effects give rise to multiple non-classical phenomena such as electron density spill-over across the surface of the nanoparticle.

We have introduced\texttt{Sapphire} as new platform from which one may analyse a classical molecular dynamics trajectory and streamline the creation of informative figures. We have motivated the need for such a platform to exist, especially given the community-wide need to have a database of nanoparticles with a common ontology. We believe that \texttt{Sapphire} has a role to play in the creation and maintenance of such a database as it has proven itself a powerful characterisation tool. More to this, we have begun to introduce pattern recognition of complex surface features into its repertoire of features. We believe that by a systematic approach of characterisation followed by classification, we may proceed to develop this desired database which may predict the properties of unseen structures. Granted this ambition is still to be realised in the future, we still maintain that we have taken the correct steps in its realisation through the development of Sapphire.

We proceeded to demonstrated the role that geometry has with respect to the optical properties of a nanocluster with respect to both absorption and emission spectra. This was achieved in two ways. By adopting a classical, atomisitc description of light-matter interactions via the use of the GDM. Using this method, we were able to demonstrate the sensitivity that classical approaches may still have to the morphology of a given nanoalloy. Moreover, by construction this method is expected to be in good agreement with larger scale experimental configurations compared to those discussed explicitly within this thesis. Furthermore, we have performed TDDFT investigations to monitor the distribution of electrons in ultra-small nanoalloys, and demonstrated how one may predict and compute optical extinction using either explicit TDDFT or by solving Casida's equations in the linear response regime. We discussed and explored common issues with both methods so as to improve the quality of subsequent calculations. In an effort to more accurately recreate experimental conditions, we have performed investigations with the laser feature in \texttt{Octopus} and have generally found reasonable agreement with respect to the well-established $\delta$-kick and Casida formalism for evaluating the dipolar response of small Jellium clusters. We have explored using this \textit{ab initio} method the effect that alloying may explicitly have on static and time dependent properties. This investigation demonstrated these variations by the comparison of two Au$_{20}$ clusters in which the large HOMO - LUMO gap of the tetrahedron had closed entirely when transitioning into a structure exhibiting C1 symmetry. By evaluating the evolution of the dipole moments in time, we were able to qualitatively discuss the possible origin of excitations visible in the aforementioned spectra. While this will require more robust analysis tools in future iterations of such studies, these techniques may certainly provide a pathway to more effectively discriminate between a localised surface plasmon and a single electron-hole pair excitation. In this chapter, we have demonstrated the strong influence that the introduction of Pt may have on the static and optical properties of small Au cluster. We have considered the characteristic of the orbitals in the region of the HOMO and the LUMO, in conjunction with the size of the gap and the presence of given optical modes. By studying these NAs at the level of TDDFT, we have been able to demonstrate how one may finely tune the optical properties, and even the localisation of potential hot carriers, by simply introducing Pt in a specific and targeted approach.

By creating composite clusters of Pt deposited onto larger Au clusters, we have created a caricature facsimile of commonly described physical systems of Au nanoparticles decorated in Pt. By recreating such structures in microcosm, we may assess their stability at finite temperature and make predictions regarding the long-time sustainability of a Pt surface available for molecules to adsorb onto. In our studies, we have determined that the more highly coordinated $(111)$ facets are more stable surfaces upon which a decoration may be deposited. In antithesis to this, it was found that by having an edge-edge conduit, one may expect rapid and total degradation of a deposited Pt nanoparticle, regardless of relative sizes. By molecular dynamics simulations, we predict the instability of Pt-decorated (about 1 nm) Au nanoparticles (between 1.5 to 4.6 nm) in vacuum. We show which styles of chemical ordering are most likely to occur as a function of the Au morphology, temperature, and Pt-loading. Our primary focus has been on low Pt-loading, less than 50\%, to keep our structures consistent with those in the experimental literature, in absence of ligands which may significantly alter the surface energy of Au and Pt. Through our metastable quasi-Janus and the most energetically favourable partial onion-shell are the most likely chemical ordering. In both cases, a Au-skin of just one layer is formed in agreement with the general trend to have a surface layer rich of the metal with lower surface energy and larger atomic radius. We also reveal  the time scale of these chemical re-ordering, being less than 100 ns at 600 K. The further diffusion and dissolution of Pt atoms inside the Au core depends on the Au morphology, with icosahedra more likely to facilitate the formation of a Pt-subshell. We investigate the atomistic  mechanism proceeding from Au adatoms which climb and diffuse onto the decorating Pt-nanoparticles and the concerted motion of Pt-diving into Au surface and sub-surface, because the formation of an Au-vacancy close to the Pt-edge. Our numerical predictions shows chemical features that can alter and even determine the lifetime of plasmonic nanocatalysts. In this respect, our predictions may guide future rational design of heterostructures made of decorating Pt-nanoparticles of Au-plasmonic cores.

Given the observations of the previous investigation, and motivated by the broad expanse of alloys utilised for photo-activated catalysis, we proceeded to undergo an intensive, systematic study of the thermal stability of eight commonly used metallic nanoalloys. We generally found, with caveats and exceptions, that there is a trend of less cohesive metallic species dephasing towards the surface of a given alloy. Often it is that the less cohesive metal is the less catalytically active specie, Au and Pt for example, meaning that by thermally activated processes it is likely that the surface of the catalyst may become polluted by a less desirable chemical specie. We have found that by increasing the relative abundance of the more cohesive component of the structure, we may too increase the stability and protect against pollution. However, this comes with the obvious drawback as such elements are often prohibitively expensive and the incentive is to use as little as feasible if such clusters are to become attractive in their uptake beyond the scope of scientific research. Nonetheless, we found that the Core-Shell configuration with a sufficiently thick interface region between the core and the surface were generally the most robust against thermally activated restructuring and even melting. We argued that this layer acted as a barrier against migrating core atoms which utilised the high cohesivity of the catalytic species to its advantage. Whilst it may be argued that such a cluster may not be globally stable, they appear sufficiently so given the temperatures our investigations subjected them to. Indeed, our reviews of the corresponding literature generally revealed that this style of chemical ordering, or iterations of it, are common practice.

Our final investigation had the scope of determining what role environmental effects may play on the stability and properties of a nanocluster, and indeed the converse, too, was considered. We began by considering the introduction of a simple implicit solvation model which was initially designed with our inter-atomic potentials as the central consideration. We explored the how this solvation model may influence the the initial chemical ordering regimes modelled for our selection of small AuPt nanoparticles. It was found that an environment which interacts strongly with Pt but weakly with Au will lead to an increased stability of the nanoalloy even at sustained high temperatures. Moreover, it was again observed that the most robust chemical ordering was that of the Core-Shell; as determined in the previous investigation. An obvious next step in this investigation is to perform follow-up \textit{ab initio} investigations to find solvents which will have the desired property. Such a study, albeit in its infancy, and incomplete, is presented at the end of this chapter. In this investigation we began to compute the binding energy of water molecules with small Pt inclusions on the AuPt nanoalloys considered in Chapter \ref{c:L-M}. These early results demonstrated that there is a bias towards more highly coordinated Pt atoms with respect to binding energy. It remains our task to further probe the configuration space to have a more complete and holistic description of the AuPt-H$_{2}$O phase space. This must be targeted and systematic in its scope, however, given the enormity of this configuration space even at the ultra-small scale. We concluded this chapter with an ultra-fast simulation of an AuPt Core-Shell structure, selected for the reasons outlined earlier, explicitly solvated in water. Whilst the dynamics were too brief to make general deductions, we were able to make observations regarding the interaction of water with these clusters. Most surprising was the dissolution of a single H atom into the interstitial region between the Pt, indicating the early stages of the catalytic reduction of water. The driving theme behind this research project.

Considering what we have learned insofar, it is clear that we have found a viable direction from which to approach the problem of designing a plasmon enhanced photo-catalyst. Specifically that we must maintain this multi-scale approach to sufficiently understand the pertinent properties of these complex nanosystems. We have demonstrated the role that geometry, size, and relative chemical composition will have on the efficacy and stability of these systems - identifying particular configurations which may either enhance or inhibit performance. With this understanding, we may continue to probe the initial question of ``What properties should a catalyst designed to split water have?'' by continuing our investigations on external environments within which we may deposit our clusters. While this may be computationally intractable at the DFT scale for the sizes we truly wish to model and at the desired timescales, we are still able to design simulations which may capture the spirit of the question we wish to answer. Nonetheless, we may continue to evaluate the adsorption energy of various site interactions by depositing water molecules onto the surface of our nanoalloys and computing binding energies to more rigorously map the interactions described in the implicit solvation model to the metal-solvent affinities determined in the \textit{ab initio} regime. It is worth remarking in the closing comments that it remains necessary to consider the trifecta of optical activity, chemical stability, and solvent interactions if we are to comprehensively answer the driving question of ``What is the best nano-architecture which enhances photo-catalytic performance?''. This sentiment has unequivocally been demonstrated throughout this thesis and at the heart of each of the investigations instantiated herein. Given the results we have presented, we have gained a sufficient understanding of the problem to ask nuanced questions to comprehensively probe the beautiful world of plasmon enhanced photo-catalysis on metallic nanoparticles. Finally, we have illustrated throughout this project the utility of bringing to bear a variety of cross-disciplinary techniques to better comprehend these highly non-equilibrium systems.
