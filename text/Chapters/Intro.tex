\noindent\enquote{\itshape I am a man with a one track mind\\
    So much to do in one lifetime\\
    Not a man for compromise and wheres and whys and living lies\\
    So I am living it all, yes I am living it all\\
    And I am giving it all, and I am giving it all}\bigbreak

\hfill Queen - I Want It All

\vspace*{0.05\textheight}

In this chapter, we \footnote{Throughout this thesis, the plural first person will be used almost exclusively. Whilst this is relatively awkward when reading, the author considers it preferable to the alternatives of the singular first person or the passive construct. Therefore, the author would like to invite the reader to consider his or herself as part of the scientific `we'. Alternatively, the may consider F.D.C Williard to be included in the plural first person.} shall introduce the project within the context of the problem we aim to investigate. That being a multi-scale approach to determining the effects that structural properties such as chemical ordering, size, and alloying composition may have on plasmon enhanced photo-catalytic performance. We introduce and describe the fundamental ingredients being considered in this project in a brief yet pedagogic fashion: multi-scale approaches, plasmonics, photo-catalysis, and metallic nanoalloys. We shall then introduce the style of techniques used to address this problem - saving technical detail for Chapters \ref{c:Theory} and \ref{c:Methods}. Following this, we shall briefly discuss recent significant results and ideas within the context on this project, identifying how and where they have helped to mould the outlook and aims of this project. Finally, we shall introduce and outline the story to be told in this thesis.

\section{Motivation}
\label{Mot}
Metallic nanoparticles are ubiquitous in today's world, finding a broad range of contemporary uses \cite{BioSensors,PlasmonSensing2021,SolarToChem}, spanning an equally eclectic range of utilities. Given the finite size of these structures, many of the standard techniques used in a traditional solid state physics education are found to be profoundly limited as we do not have the luxury of an infinitely repeated unit cell or the Bloch theorem. As such, it has only truly been in the last 30 years, as computational techniques have improved and more sensitive experimental equipment have become more common, that the field of nanotechnology has really flourished. To be specific with nomenclature, the International Organisation for Standardisation defines a nanoparticle to be a discrete object whose dimensions are less than 100 nm in all three Cartesian coordinates.

Society's interest in the strong coupling between nanoscopic metallic structures and light is many thousands of years old with examples of such structures found to present magnificent properties in ancient artefacts demonstrating the antiquity of the interest. In the past 200 years, this fascination has attracted the eyes of the scientific community - followed shortly after by tangible technological innovation. Today, metallic nanoparticles have drawn wide attention as catalysts offering novel properties. A monstrous problem to consider at the nanoscale is that of characterisation. Fortunately, there are many experimental techniques such as spectroscopic and microscopic methods which may be employed in tandem with the theoretical techniques more broadly discussed in this thesis. Through these manifold characterisation methods, one is able to probe both the form and function of the metallic nanoparticles tailored to be catalysts.

A specific advantage in using nanoparticles over extended surfaces for catalytic applications is the high surface to volume ratio observed at the scale - increasing the ratio of potential active sites to 'inert' support. Moreover, the complex geometry observed at the nanoscale provides manifold adsorption sites which vary with respect to structural properties such as atomic coordination number and morphological classification such as FCC or BCC. 

Broadly speaking, there are two schools of thought governing the physical realisation of nanoparticles. To either make something large smaller, top-down methods; or to make something small larger, bottom-up methods \cite{Fra_Review}. This simplicity in description belies the actual complexity of the problem, and while we shall not detail all of the techniques here, we shall instead acknowledge some of the more prominent methods and guide the reader to more esoteric literature. In the case of the former; common techniques include dry and chemical etching \cite{Etching}, milling \cite{AuMilling}, and lithography \cite{AuLithography}. These techniques are often very energetically expensive and it can be difficult to achieve mass production quantities with a top-down approach. Alternatively, the bottom-up methods allow for one to access a wide range of constituent components from which to assemble their systems. Typically, these methods will be either a co-precipitation which is commonplace for Au nanoaparticles \cite{AuCoprecip}, or techniques such as molecular beam epitaxy \cite{AuEpitaxy}. An advantage to these methods is that they are often energetically cheaper to perform and provide a higher degree of control over tailoring specific properties of the nanoparticles. Following creation, it is common for samples to be deposited onto a substrate grid and then studied under ultra-low pressure conditions.

Furthermore, by simply extending from monometallic clusters into the domain of bi-metallic alloys, one opens up a wide boulevard of not only morphological properties, but also the impossible to overstate importance of selectively choosing metals for specific properties. This is because one may select metals in a fashion which directly improves the selectivity of a catalyst, its activity, or even prolonging its life by making it more resistant to poisoning due to electronic or geometrical effects. As an example. IrPd nanoalloys have been studied due to their increased performance for catalysing the reduction of CO over Ir or Pd alone. Moreover, by alloying expensive and rare transition metals such as Pt or Rh with less valuable and more abundant metals such as Cu or Ni, one may effectively reduce the absolute cost of synthesis. By taking advantage of and further studying these properties of nanoalloys, one may help to bridge the chasm between scientific enquiry and broad industrial adoption. As an example, Rh has been widely studied as one the most effective metals for both reduction and oxidation reactions, however its scarcity has prevented it from being more widely adopted industrially. There is now a growing interest in alloying Rh with other metals for the purposes discussed above.

It cannot be overstated how crucial morphology and chemical ordering (the fashion in which the two metals are brought together) are with respect to catalytic activity and nanoparticle stability. Indeed, we shall discuss in detail the latter, as the intentions of fabricator and physics may not necessarily be in harmony. That is to say that a chemical ordering favourable for a designated chemical reaction may be metastable at best or worse, grossly unstable. Therefore, it is crucial to understand the structural and chemical stability of these nanoalloys if one is to model and design effective syntesisable catalysts. However, the sample space for nanoalloys is far beyond vast, even if one constrains one or more structural parameters such as size, chemical ordering, or relative abundance of the two constituent species. Consequently, a brute force strategy to describe the phase space is optimistic bordering on impossible.

It is in this vast nanoworld that we conduct this research - a collection of projects focused on utilising methods across a range of time and size scales with the purpose of shedding further light on the hot topic of plasmon enhanced photo-catalysis on metallic nanoparticles. Through this introduction, we shall discuss the context in which this problem exists, identifying ongoing expeditions into the nanoworld, and present the broader structure of how this thesis shall be presented. It is hoped that upon having read this thesis, one shall have a deeper appreciation as to how much room there still is a the bottom, and the true beauty of finite nanoparticles.

%%%%%%%%%%%%%%%%%%%%%%%%%%%%%%%%%%%%%%%%%%%%%%%%%%%%%%%%%%%%%%
Given current rising concerns as to the sustainability of current sources of energy production, it is critical to provide alternative means of generating and storing energy. It has been hypothesised that the production of molecular hydrogen, generated from the splitting of water \cite{H2Dissociate,MoreH2Dissociate}, may be used for hydrogen fuel cells - an alternative to many other contemporary batteries. Moreover; there are parallel studies which seek to catalyse the reduction of CO$_{2}$ into new hydrocarbons on Cu surfaces \cite{LSPCatalysis}. These studies, in principle, take advantage of the same physics discussed here. As such, it may be argued that a profound, fundamental understanding of the underlying physics behind plasmon induced photo-catalysis is of key concern to the modern scientist - as are the technologies unlocked by these physical principles which may offer viable routes to avoid the impending antrhopogenic climate crisis.

Through the utilisation of localised surface plasmon resonance, it has been shown that various chemical reactions may be catalysed via the mechanism of hot charge carrier injection from a given metallic nanoparticle into an adsorbed molecule. In Section \ref{sec:plasmons} we provide a more detailed and quantitative discussion of these phenomena - for now we shall present qualitatively the significance of this mechanism.

In general, it is largely accepted that many fundamental and industrial reactions may be more efficiently catalysed via the use of plasmon enhanced photo-catalysis over alternatives such as thermocatalysis \cite{ZHAO2019920,SAMANTA2017621}. Indeed, there is an understanding that not only is the efficiency of the reaction improved, but so too are the activity, selectivity and even stability of the catalyst itself. Typically, these chemical reactions are accelerated either vibrationally through the direct injection of kinetic energy into the adsorbed molecule via the transfer of kinetically charge carriers; or alternatively via the introduction of the charge itself to facilitate the accelerated process. Moreover, the vast configurational space available to the aspiring theoretician with respect to composition, morphology, and chemical ordering it is clear that, to quote Richard Feynman "There is plenty of room at the bottom."

\section{Multi-scale approaches}
\label{sec:multiscale}

In general, one is interested in physical phenomena which may occur at a range of time and length scales. For each scale, we will usually be equipped with no fewer than one standard model and framework to treat such systems. For example, the macroscopic phenomenon of field enhancement around an optically active nanoparticle may be sufficiently treated with finite element methods \cite{Girard_2005}. However, when considering ultra-small systems where quantum mechanical properties dominate, it is necessary to adopt an appropriate model to treat the dynamics and interactions \cite{Strongfieldlightdressedsolids}. Herein lies the core principle behind multi-scale approaches. Either we are limited by computational constraints - a quantum mechanical description is computationally prohibitive for macroscopic aperiodic systems - or the properties we wish to investigate are beyond the reach of 'large picture' models. An issue remains. Sometimes we require both the scope of a more general view in conjunction with the precision of an \textit{ab initio} model to have a sufficient handle on the nature of a system under scrutiny. Wherein lies the issue is how to reconcile these two models as they are, in principle, describing the exact same properties meaning that they should completely coincide at some specific point. However, it is not common for this point of coincidence to necessarily exist within the domain of the adjacent regions of the multi-scale hierarchy. 

Nonetheless, it is not uncommon to consider separately these regions described by characteristic time and length scales \cite{C5CP01096A,Coal,MATOUS2017192}, and consolidate the findings into a more cohesive and heuristic description of the system under consideration. Indeed this is the approach we take in this thesis. Treating separately the opto-electronic response of ultra-small systems and the dynamical rearrangement of mesoscale nanoalloys by respectively using \textit{ab initio} approaches and classical molecular dynamics.

\section{Existing research}
\label{sec:Res}

Au-based nanoalloys, where the Au seed is much larger than the catalytic metal, \textit{e.g.,} Pt, decoration, are the subject of great interest in recent studies. It is understood that the plasmonic properties of the Au seed operating in conjunction with catalytic properties of the other metal may yield a highly efficient photocatalyst \cite{Cortes2020,Zhang2019}.
Among the various catalytic metals to work jointly with Au, Pt has been widely studied. In particular, a lot of attention has been dedicated to Au-core and Pt-shell heterostructures, which have enhanced photocatalytic performance \cite{Adzic2013,Xia2017}. 
These hetereostructures are obtained decorating large Au seeds of about 12 $\pm$ 2 nm in diameter with smaller Pt clusters with a diameter of about 2 $\pm$ 1 nm \cite{Kunwar2019,Engelbrekt2021,Linic2021,Fagan2021}. The coinage metal, as Au, is used as a reservoir of hot electrons, accessible to the catalytic component, as Pt \cite{Xia2017}. Such nanoalloys are often characterised by a large optical response before the energy is thermalised into vibrational modes. The thermalisation through electron~-~phonon coupling seems to depend on the morphology of the nanoparticle \cite{Staechelin2021}. The formation of electron~-~hole pairs occurs in specific locations within the nanostructure, likely at the plasmonic (Au)/catalytic (Pt)interface. Recent atomistic simulations revealed that the spatial distribution of hot electrons in silver nanoparticles depends on the morphology and the local site \cite{TRossi2020}.
The mechanisms to the flow and extraction of energy and charge carriers, how the plasmon is distributed, where it localises, and its temporal evolution are open challenges in photo(plasmo)catalysis \cite{Linic2021}.

The relative chemical composition of the nanoalloy surface affects strongly the absorption spectra of the AuPt core-shell \cite{Rocha2021}, while the relative distribution of chemical species within the core influences less the optical properties of the nanoalloy \cite{Stener2021}.
Hence an open challenge is to understand the chemical and structural stability of the decorating catalytic clusters onto a Au seed. Indeed, the Pt loading and shell-thickness of the catalytic material can hinder hot carriers generated in the Au-core to reach the surface \cite{Jorge2021}.
Furthermore, a significant drop of the catalytic performance is likely due to the disappearance of Pt active sites at the surface \cite{Jorge2019}.

The chemical ordering of AuPt nanoalloys has been studied mainly on truncated octahedra, thus respecting their bulk FCC layering, and in vacuum \cite{Divi2016, Jagannath2018,Stener2021}. There is a strong evidence that Au atoms surface segregate. This behaviour is in very good agreement with the general trend observed in metallic nanoalloys, where the chemical species with lower surface energy, and larger atomic radius stay preferentially at the nanoalloy surface \cite{Namsoon2021}.

Recently, there has been great attention paid to the role that chemical ordering may play in the stability \cite{Coal2}, and plasmonic characteristics of a nanoparticle. In the case of coalescing multiple nanostructures, two or more nanoparticles collide and begin to merge to form a single, larger aggregate. This phenomenon is driven by the minimisation of surface area and the resultant decrease in surface energy that occurs when two smaller particles form a single aggregate, whose surface to volume ratio will be finally smaller than that of the initial colliding units. Naturally, this is a highly non-equilibrium process, and therefore it is difficult to predict the dynamical behaviour of particles undergoing this process. By studying the role that chemical ordering may have on the re-equilibration of a composite nanoparticle, we may begin to identify themes which lead to the design of more stable systems. It is necessary to have a deep understanding of theoretical and experimental descriptions of the coalescence phenomena at the nanoscopic scale if we are to identify the qualities pertinent to the stability of catalytic nanoparticles. This line of thinking motivates our design philosophy at the classical level of theory: to understand better the thermodynamics of coalescing nanoparticles. 

Indeed, we must also consider the consequences that chemical ordering may have on electronic and plasmonic properties, as discussed in \cite{Ranno2018}. In that simply the order in which metallic species appear seems to influence the calculated properties by re-configuring the electronic wavefunctions both deep within the cores of the clusters and at the surface. These results and analyses have motivated our own studies at the quantum mechanical level of theory in that we also wish to calculate the effects to ground and excited state properties when the species constituting a core-shell cluster are inverted. 

High-resolution transmission electron microscopy (HRTEM) shows the formation of a Pt shell, up to 1-2 layers thick, grown epitaxially onto Au and hence preserving the Au faceting \cite{Aidan2017}.
More and more sophisticated TEM tools boost the numerical search of study aggregation/coalescence of bimetallic systems, where two monometallic nanoparticles, of different materials, hit each other \cite{Nelli2021}.

Besides techniques to stabilise a Pt-skin onto an Au-core, choosing specific environment which lowers the Pt-surface energy \cite{Bian2015}, there is experimental evidence that the Pt-decoration(s) are not stable on Au-core and they luckily move underneath Au or coalescence, confirming the best chemical ordering predicted for AuPt \cite{Hong2019, Guo2020}. However, very few calculations have addressed the kinetic rearrangements in AuPt nanolloys in vacuum, but only limited to few cases, in terms of sizes, shapes, and chemical composition \cite{Jagannath2018,Chen2021}.

While we might expect that Pt-decoration are not energetically stable, no studies reveals the timescale or the atomistic mechanisms for chemical reordering, depending on Au core morphology and Pt loading.

For probing the population and distribution of hot carriers within a nanostructure, it is common to consider the ultrafast dynamics of photoexcited electrons in the plasmonic metal. This indeed is a key process underpinning various practical applications beyond the specified utility of plasmon enhanced photo-catalysis \cite{Brongersma2015,HotImpossible,Clavero2014}, as discussed in this thesis, where alternative uses range from ultrafast switching \cite{ControlPhase,Karnetzky2018,Hendrick2017} to sub-band photodetection \cite{HollowMetal,Knight2011,Zhou2019}.

As discussed above, the field of research regarding photo-catalysis on nanoparticles alone is vast and while we do not wish for this to become a review article, it is only appropriate to discuss recent contributions which have profoundly moulded the direction taken within this project. Within the scope of this project, we shall discuss a wide range of techniques employed to model the hot carrier creation and transfer on a nanoparticle's surface \cite{NordLander,HotCarrierTransport}. A topic in which there is consistent development and advancements from the side of computing hot electron populations and their dynamical evolution. 

Also to be considered are the advanced sampling techniques for finding thermodynamically stable clusters \cite{Structure_Thermo_EnergyLandscape,FerrandoEquilibrium}. Given the large phase space in which nanostructures exist, it is a highly non-trivial problem to find energetic minima, even should one elect to be constrained to only a handful of collective variables when describing the state of their nanoparticle. This problem is further exacerbated when considering complex alloying configurations, the interaction with a substrate or support, or the interaction of a nanoparticle with molecules and ligands. Considering that physical realisations will almost invariably consider several of these obfuscating factors, it becomes clear to see why efficient sampling methods are necessary to navigate the increasing complex phase space inhabited by these beautiful objects. 

We also acknoweldge the growing trend in employing machine learning techniques in the development of atomistic forcefields for molecular dynamics simulations \cite{MLInterface,hansen2019atomisticMLFF,ClaudioMFF,Flare} for increasing the size and time scales accessible by such models. It is not simply in the domain of molecular dynamics in which there is an increasing appetite for machine learning techniques to improve the efficiency of common numerical approaches. \textit{Ab initio} methods have also flourished with the new age of utilising machine learning to either solve intractable problems or to increase the efficiency of pre-existing methodology \cite{PhysRevResearch.4.023126,PhysRevX.10.041026,doi:10.1063/5.0024570,Nagai2020,doi:10.1126/science.abj6511,C8RA07112H,Schleder_2019}. Whilst we do not explicitly consider such methods in the following work, we acknowledge that machine learning techniques are quickly becoming a popular and ubiquitous instrument of the modern quantum chemist and material scientist. Indeed, Machine learning has indeed been a controversial point of discussion within the physical sciences community. With some sceptical about the reliability of solutions, and others heralding its introduction as a new epoch in material modelling.

While it is well established that there must be a hot carrier transfer process from catalyst to molecule \cite{doi:10.1021/acs.chemrev.7b00086,10.1039/9781839164668-00172}, it is not necessarily well understand how this process develops at the electron scale. Moreover, it is often difficult to determine whether or not charge has been transferred directly, in that an electron from within the Fermi sea of the catalyst may be transferred to an unoccupied state of the adsorbed molecule - or vice versa. This is known to be a more efficient charge transfer mechanism that that of indirect transfer, in which an electron - hole pair is created within the catalyst and then a hot carrier is transferred to the molecule. This is inherently less efficient as it is more likely that the hot carrier pair will recombine in this regime. To study this effect at the atomistic scale, a recent paper has been published in which a single CO molecule had been adsorbed onto one of three unique bonding sites of an Ag$_{147}$ nanoparticle \cite{AuTRansfer}. It was determined that the geometry of the adsorption site was significant in promoting the direct charge transfer mechanism. From this study, we may follow a similar line of questioning to determine how best to adsorb water onto the surfaces of our composite AuPt nanoparticles.

We acknowledge the extensive research conducted on the creation of metallic NAs through various doping processes for a variety of novel purposes, from enhanced catalytic properties \cite{doi:10.1021/jp070791s,C7NR05871C} - to nurturing antimicrobial characteristics \cite{PATHAK2019e01333}. Moreover, we acknowledge the work of Lozano \textit{et al.} \cite{HWC} in performing pioneering studies on the doping of Au structures with Ag to determine similar optical properties presented in this study. However, we are not aware of any literature relating Pt doping in Au nanostructures in the fashion we propose. As such, we believe that this work will help to develop a more fundamental understanding of larger nanostructures which are currently being investigated and deployed as plasmocatalysts \cite{JorgeStructure}.

In addition to the use of AuPt nanoalloys for catalytic and sensing processes, there also exists a broad span of literature on alternative compositions of nanoalloys whose appearance and prevalence in both the numerical and experimental communities demonstrates their wide-reaching utility. We shall provide a brief discussion of the literature regarding the alloying of group 10 and group 11 metals in addition to a few notable group 11 with group 11 examples such as AgAu alloys. 

Indeed, AgAu nanoalloys are widely studied, being attractive given their degree of chemical mixing influencing their optical properties \cite{C7NR05871C,AgAuNanoparticles,C0CP02845B,Ma2011,doi:10.1021/jp806930t,doi:10.1021/nl0100264,doi:10.1021/jp002438r}. It has been established that the relative charge disparity between Ag and Au is the causal factor leading to the dephasing of Ag and Au with the latter often migrating to the surface \cite{doi:10.1063/1.1492800,doi:10.1021/jp034826+,doi:10.1063/1.2210470}. As discussed previously, both Ag and Au are famous for their plasmonic character and strong optical properties in the UV-Vis range of the light spectrum. Moreover, even Au alone is noted for its catalytic properties, principally for the water gas shift reaction \cite{catal8120627,doi:10.1021/jacs.8b08246,Carter2022-bn}. 

AgCu alloys also appear profoundly across a range of uses and investigations \cite{C5CP00782H,doi:10.1021/acs.inorgchem.9b02172} for purposes such as the oxidation of ammonia \cite{doi:10.1021/acsami.9b16349}, their excellence as a model system to investigate phase stability of nanoalloys from a non-equilibrium experimental perspective \cite{SHENG2002475,MISJAK20104247,doi:10.1063/1.354582,UENISHI19911342}, and their ubiquity in use as braze alloys for joining various other metallic species \cite{ASTHANA2008617,XIONG20101096}. 

With respect to alloying plasmonic species, it has also been common to consider the mixing of Au and Cu as catalysts built from these nanoalloys have recently attracted interest for facilitating the catalytic oxidation of CO, benzyl alcohol, and propene \cite{doi:10.1021/cm302097c,doi:10.1021/acsanm.9b00904,D1AY01942B,LLORCA2008187,B817729P,B804362K,DELLAPINA2008384} in addition to having utility in the oxidation of methanol for the production of hydrogen fuel cells \cite{CHANG200955}.

AgPd nanoalloys enjoy a similar representation in both the computational and experimental literature as AuPt alloys as their composition enables their utility as nano-catalsyts for hydrogenation, in fuel cells, and as sensors \cite{Yang2021,ZamoraZeledn2021,Tsuji2014}.

AgPt nanostrucutres occupy a similar role with respect to utility as Au-Pt structures \cite{Li2013,doi:10.1021/acs.nanolett.6b03302} where the trade-off considered is the relative affordability of Ag compared to Au. However, the relative inert nature of Au compared with Ag renders Ag based catalysts more susceptible to poisining - affecting their longevity. Nonetheless, the consideration of the synthesis of AgPt nanoparticles has gained traction when considering the cost of creating the nanoparticles to form reactors in Pt based catalysis.

AuCu nanoparticles have attracted attention in recent years as biosensors for various organic molecules, such as glucose\cite{Liu2021}, Codeine, and Acetaminophen \cite{Allahnouri2022}. Moreover, given that both Au and Cu are optically active in the UV-Vis range, they are also commonly studied for their tunable plasmonic properties \cite{doi:10.1021/jp107637j}.

AuPd nanoparticles serve a similar purpose in the experimental and ocmputational communities as AuPt nanoalloys do, for their similar properties of being excellent catlaytic alloys for the oxidation of primary alcohols, or direct synthesis of H$_{2}$O$_{2}$ amongst other catalytic reactions \cite{Zhu2019,Enache2006,Zhang2011,Agarwal2017}. As has been previously discussed \cite{Zhan2011}, the catalytic performance of Pd appears to be enhanced when alloyed with Au, and therefore, this particular alloy forms the basis for another promising candidate from the perspective of catalysis. 

CuPt nanoalloys are once again known for their catalytic properties with respect to the hydrogenation of organic molecules commonly used within the feedstock industries \cite{Taylor2021,Shiraishi2013}. Consequently, they too have recently received a great deal of attention across communities for their strong catalytic properties and their structural properties when alloyed \cite{Zheng2013,https://doi.org/10.1002/chem.201905672}. Moreover, due to the large surface to volume ratio of metallic nanaoparticles, one may effectively increase the relative catalytic activity of their valuable metals by re-purposing the same volume of material from an extended slab into a collection of nanoparticles \cite{doi:10.1021/acs.chemrev.8b00696,doi:10.1021/acs.chemrev.9b00220}. However, the size of the nanoparticle itself has been demonstrated to be linked with varying selectivity of catalytic pathways, and measurable variances in the binding affinity for both the products, and the substrate itself \cite{doi:10.1021/ja904307n}.

Hence, it is clear that the modern age of highly efficient catalytic materials is driven strongly by the desire to use small metallic nanoparticles, whose performance when alloyed is far greater than when they are simply monometallic as has historically been the case within the catalytic community. However, the challenge then becomes the optimum alloying regime to select for a given structure to maximise its lifetime, minimise its cost, and boost performance. This is a challenge which requires the collaborative efforts of communities with expertise in fabrication, characterisation, catalysis, and optics. Given the objective to bring sustainable, green, chemistry to the forefront of modern industry, the span of the current literature and fervent interest in this domain is certainly justified and worthy of continued research.

%%%%%%%%%%%%%%%%%%%%%%%%%%%%%%%%%%%%%%%%%%%%%%%%%%%%%%%%%%%%%%
%%%%%%%%%%%%%%%%%%%%%%%%%%%%%%%%%%%%%%%%%%%%%%%%%%%%%%%%%%%%%%

\section{Scope of this project}
\label{sec:scope}

Primarily, this research project is focused on determining properties of nanoalloys which are considered to be candidates to photocatalyse the water splitting reaction on Au supported Pt nanoparticles through the utilisation of the intense optical and plasmonic properties of Au; with the initial design inspired by recent experiments conducted by Salmon \textit{et al.} \cite{JorgeStructure}. The rationale behind this construction being that the Pt may be described as the `fast' component of the cluster, whereas the Au may be described as `hot'. Respectively, this is because Pt is known to be an excellent catalyst for the water splitting reaction \cite{PtCatalyst}; and that Au has been demonstrated to have strong plasmonic behaviour at incident light wavelengths observed in solar output spectra \cite{AuTRansfer,SolarToChem}. It is common practice for the excited hot carriers, produced through the non-radiative damping of the localised surface plasmon generated by the Au component of the nanocatayst, to be considered to be excited via transient optical process. For example, the illumination under laser stimulation or exposure to solar radiation.  

Transient optical processes, such as laser excitation within an experimental configuration or by perturbation in most numerical realisations, are crucial when studying photocatalysis since the dynamics and the photoexcited hot-carrier population has been shown to correlate with photocatalytic performance \cite{nano9020217}. Given that an element of this project is concerned with the control over the hot-carrier dynamics; achievable either by geometrical means, or by choosing materials with different electron–electron scattering rates; we shall be heavily leaning into studying precisely these properies. That is to say, by exercising control over the morphology and chemical ordering of a given nanostructure, we may be able to directly influence the population of hot carriers within such systems in addition to influencing the candidate structure's efficacy as photo-catalyst. It is known that the size of nanostructures and its anisotropic shape influences  the  nonequilibrium  and equilibrium dynamics of the electron and hot carrier populations \cite{AnatolyGeometry,Nicholls2019,PhysRevB.69.195416,PhysRevLett.85.2200}. 

This sentiment motivates our investigation into the opto-electronic properties of nanoalloys, which we present in Chapter \ref{c:L-M} wherein our objective is to determine the influence that size, morphology, and chemical ordering may have on the calculable properties such as electron distribution and photo-extinction spectra. Both of these are considered to be indicative of photo-catalytic efficacy.

In general, the methods by which we conduct this work may be considered to be those of the broader group:

\begin{itemize}
    \item We model and design materials at the nanoscale with a focus on their structural evolution and transformations, nanothermodynamics\footnote{``Nanothermodynamics'' may appear to be an oxymoron as the thermodynamic limit required for describing the homogeneous behaviour of large systems is clearly not reached for the small systems considered in the following thesis. Rather, this is a beautifully diverse field over 60 years old \cite{10.1063/1.1732447} which continues to be developed and formalised \cite{e17010052,C4SC00052H}. In section \ref{sec:nanotherm}, we shall present a more detailed and formal description of the discipline.}, formation process.

    \item We apply numerical tools to elucidate the structure-property relationship and how shape of individual and assembled structures influences their catalytic, magnetic, optical properties.
\end{itemize}


In the following, we shall discuss both nanoparticles and localised surface plasmons, demonstrating the utility of the overlap between the two. We shall then proceed to introduce and motivate the use of density functional theory methods, emerging from quantum mechanical theory; and classical molecular dynamics, a powerful simulation tool, to model systems existing at the nanometric scale. Indeed, we intend to highlight the necessity of adopting a multi-scale modelling approach when  simulating the amazing properties of metallic nanoparticles, a fundamental consideration in this project. We must adopt a systematic approach when describing metallic nanoparticles. 

With the current state of the art established, the latter three sections of this report will focus on our contributions to this vibrant field of study. We introduce Sapphire, a purpose built platform for designing and analysing metallic nanoparticles which is fast evolving into a dynamical classification tool, whose primary successes include the rapid analysis of multiple characteristics of nanometric structures and of complex surface pattern recognition. We shall then discuss our classical simulations whose stated purpose has been to assess the stability and longevity of composite systems in which a small Pt decoration has been deposited onto a larger Au cluster. Finally, we shall discuss our quantum mechanical calculations of both ground state and excited state properties across a range of geometrical, and chemical isomers. 

Essentially, we are asking the question ``How much do shape, size, and composition matter when designing a photo-catalyst?''
%%%%%%%%%%%%%%%%%%%%%%%%%%%%%%%%%%%%%%%%%%%%%%%%%%%%%%%%%%%%%%
%%%%%%%%%%%%%%%%%%%%%%%%%%%%%%%%%%%%%%%%%%%%%%%%%%%%%%%%%%%%%%

\section{Thesis Outline}
\label{s:Outline}

The remainder of this thesis is organised as follows. 

\begin{description}

  \item[Chapter \ref{c:Theory}] provides background information relevant to the field of research. We present the general premise of metallic nanoparticles, the science of plasmonics in both classical and quantum mechanical descriptions, the physics of how plasmons may be used for photo-catalytic purposes, and how metallic nanoparticles may be utilised for the purposes of plasmon enhanced photo-catalysis.

  \item[Chapter \ref{c:Methods}]  details the numerical and analytic methods used to perform the studies carried out in pursuit of this research goal. We present and discuss density functional theory and its time dependent realisations, molecular dynamics from both the \textit{ab initio} and classical perspectives, and theoretical spectroscopic techniques using both classical and quantum mechanical descriptions.
  
  \item[Chapter \ref{c:Sapphire}] introduces the \texttt{Sapphire} library which has been used extensively to evaluate properties of metallic nanoparticles. We introduce the rationale behind developing this library, and present examples on how it may be used.

  \item[Chapter \ref{c:L-M}] is dedicated to the study of light-matter interactions with respect to metallic nanoparticles. We shall present techniques from \textit{ab-initio} to classical and draw comparisons between both. Given the nature of this project, understanding how light may interact with nanoparticles is necessary to meeting our research goals.
    
  \item[Chapter \ref{c:Coal}] details our study into the simulating of various metallic nanoparticle formation processes primarily \textit{in vacuo} and present an alternative approach to modelling an implicit interacting environment at the classical level. We take inspiration from existing experimental work in an effort to model these dynamic processes. 

  \item[Chapter \ref{c:Alloy}] presents the structural and morphological properties of various nanoalloys undergoing a rapid melting followed by an annealing process to assess dynamically favourable conformations dependent on the initial chemical ordering and species present.

  \item[Chapter \ref{c:Water}] provides a discussion on two ways in which one may introduce an interacting environment to a nanosystem to assess how this environment may alter the morphology or electronic structure.
  
  \item[Chapter \ref{c:Conclusions}] concludes this thesis. We shall present the core principles of the research carried out - re-establishing the central narrative of the beauty at the nanoscale. Following this, we shall illuminate interesting aspects which require a broader scope with respect to the work to appreciate sufficiently. Finally, we shall provide a thorough discussion of future research which may arise from our findings and investigations herein.
  
\end{description}

