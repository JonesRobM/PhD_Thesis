
Metallic nanoparticles have long been studied for their vast array of promising, novel applications; from biosensors to photovoltaics. We are particularly interested in the
optical properties of noble metal metallic nanoparticles, primarily gold and platinum, as a means of probing their viability as photocatalysts for the splitting of water into its molecular components to facilitate the extraction of molecular hydrogen to potentially be used in fuel cells. Particular attention is paid to noble metals as they exhibit surface plasmon resonance in the Ultra-violet-visible (UV-Vis), making them ideal candidates for use in solar technologies. Furthermore, we aim to characterise and classify the geometrical properties of these metallic nanoparticles to identify the impact that local properties may have on optical responses. Moreover, we suspect that such properties may have a profound influence on the location and width of the localised surface plasmon resonance peak; hence our investigations are probing these structural parameters.
%
We perform simulations across a range of time, length, and theory scales to develop a deep understanding of the intimate interplay between structure and properties with respect to metallic nanoparticles. In this endeavour, we have adopted the standard techniques of molecular dynamics to probe cluster dynamics on time-scales approaching the microsecond and length scales approaching and exceeding 10 nm; and time dependent density functional theory to probe the ultra-fast electron dynamics within an electron system far from equilibrium.
%
Finally, we present an original contribution to the nanoalloy community with the Sapphire package. A self contained python package which provides a wide array of structural analyses tools with the purpose of characterising and classifying nanoalloys as core feature of its design philosophy.
%
In this thesis, we present the structure-properties relationship in detail, using the tools described above; and we attempt to determine the necessary structural features and properties that a NA catalyst should be designed to have.
%